\documentclass[a4paper,12pt]{article}
\usepackage [utf8]{inputenc}
\usepackage [italian]{babel}
\usepackage{graphicx}
%\usepackage{grffile}
\usepackage{listings}
\usepackage{color}
\usepackage[hidelinks]{hyperref}
\usepackage{calc}
\usepackage{caption}

\addtolength{\oddsidemargin}{-40pt}
\addtolength{\textwidth}{80pt}
\addtolength{\voffset}{-60pt}
\addtolength{\textheight}{100pt}


\graphicspath{ {../Schema concettuale} {../Schema logico relazionale} {../Piani di accesso} }
\lstset{inputpath=../Sorgenti SQL/}

\definecolor{codegreen}{rgb}{0,0.6,0}
\definecolor{codegray}{rgb}{0.5,0.5,0.5}
\definecolor{codepurple}{rgb}{0.58,0,0.82}
\definecolor{backcolour}{rgb}{0.95,0.95,0.92}

\lstdefinestyle{mystyle}{
    language=SQL,
    backgroundcolor=\color{backcolour},   
    commentstyle=\color{codegreen},
    keywordstyle=\color{magenta},
    numberstyle=\small\color{codegray},
%    xleftmargin=-10px,
    stringstyle=\color{codepurple},
    basicstyle=\ttfamily,
    breakatwhitespace=false,         
    breaklines=true,                 
%    captionpos=b,                    
    keepspaces=true,                 
    numbers=left,                    
    numbersep=10pt,                  
    showspaces=false,
    showstringspaces=false,
    showtabs=false,
    tabsize=4
}
\lstset{style=mystyle}

\addto\captionsitalian{%
\renewcommand{\lstlistingname}{Codice}}
\addto\captionsitalian{%
\renewcommand{\lstlistlistingname}{Elenco dei listati di codice}}


%%%%%%%%%%%%%%%%%%%%%%%  FRONTESPIZIO %%%%%%%%%%%%%%%%%%%%%%%

\title { \vspace{-1.0cm}{\small Università di Pisa\\Dipartimento di Informatica\\Corso di Laurea in Informatica\\[0.5cm]Corso di Basi di Dati (244AA), prof. Giorgio Ghelli\\[0.7cm]}Progetto ``Studio professionale fatture"\\Relazione finale }
\author { Candidati:\\Alessandro Antonelli\\(matricola 507264, corso A)\\Tony Agosta\\(matricola 544090, corso A)}
\date { Consegna: 25 marzo 2021\\Appello straordinario marzo 2021\\A.A. 2019/2020 }

\begin {document}
 \maketitle
 
 \tableofcontents

\listoffigures

\lstlistoflistings

 \clearpage
 
%%%%%%%%%%%%%%%%%%%%%%%  CONTENUTO %%%%%%%%%%%%%%%%%%%%%%%

 \section{ Descrizione del dominio }

Uno studio professionale, inteso come studio di un commercialista, si occupa di gestire le pratiche in corso intestate ai suoi clienti, i quali presentano fatture di cui devono essere aiutati a pagare le tasse.

I clienti dello studio sono suddivisi in due sottoclassi partizione, Organizzazioni e Persone; le Organizzazioni possono essere titolari di 0, 1, o più pratiche, mentre tra le Persone, alcune sono titolari di 1 o più pratiche, altre non sono titolari di pratiche e ne seguono 1 o più per conto di una o più organizzazioni all’interno delle quali ricoprono un ruolo, tenendo conto che un ruolo può essere ricoperto da una o più persone, che un ruolo corrisponde a una e una sola organizzazione e che a una organizzazione corrispondono 1 o più ruoli.

Per ogni cliente che si rivolge allo studio viene aperta una pratica, a cui corrisponde un insieme di fatture. Questo vuol dire che una pratica, per esistere, deve essere intestata a un cliente ma un cliente può non avere pratiche in corso intestate in quel momento, in quanto alcuni clienti non sono titolari di pratiche ma possono seguirne una per conto di un’organizzazione per cui lavorano. Inoltre, ad ogni pratica possono essere associate più fatture e viceversa una fattura può essere relativa ad una sola pratica.

Una fattura può essere pagata con modalità di pagamenti differenti e a rate, questo vuol dire che un pagamento si può riferire ad una sola  fattura, ma una fattura può essere pagata, fino a raggiungere l’importo totale della fattura, da più pagamenti. A tal fine un cliente può effettuare più pagamenti, e un pagamento può essere effettuato da un solo cliente.

Ogni fattura è intestata a un solo cliente, e un cliente può avere 0, 1, o più fatture intestate. Un’organizzazione emette una o più fatture, e una fattura viene emessa sempre da una sola organizzazione.

Le classi individuate sono  7:
\begin{enumerate}
\item \textbf{Clienti}: rappresenta l’insieme di tutti i clienti dello studio professionale con le loro informazioni comuni.
\item \textbf{Persone}: rappresenta l’insieme delle persone sia come clienti dello studio professionale che, come persone, facenti parti di  una organizzazione. Specializza la classe “Clienti”.
\item \textbf{Organizzazioni}: rappresenta l’insieme delle organizzazioni clienti dello studio professionale. Specializza la classe “Clienti”.
\item \textbf{RuoliAziendali}:rappresenta l’insieme dei ruoli svolti da ogni persona che fa parte di una organizzazione.
\item \textbf{Pagamenti}: rappresenta l’insieme dei pagamenti con tutte le relative informazioni.
\item \textbf{Fatture}: rappresenta l’insieme delle fatture, con tutte le relative informazioni.
\item \textbf{Pratiche}: rappresenta l’insieme delle pratiche associate ai clienti.
\end{enumerate}

 \section{ Schema concettuale }

\begin{minipage}{\textwidth}
\begin{center}
\centering 
 \captionof{figure}{Schema concettuale a oggetti}
\centerline{
%\frame{
\includegraphics[width=\textwidth]{ Schema concettuale a oggetti.png }
}
\end{center}
\end{minipage}

 \subsection{ Vincoli }

\subsection{ Vincoli intrarelazionali }

\begin{itemize}
\item Tutti gli attributi (comprese le chiavi esterne) hanno il vincolo NOT NULL.

\item Il \textit{Nome} e il \textit{Cognome} di una \textit{Persona} deve essere lungo almeno 1 carattere.

\item Il \textit{CodiceFiscale} di una \textit{Persona} deve essere lungo esattamente 16 caratteri.

\item L'\textit{Importo} di una \textit{Fattura} deve essere $> 0$.

\item La \textit{CifraPagata} di un \textit{Pagamento} deve essere $> 0$.

\item La \textit{PartitaIVA} di un'\textit{Organizzazione} deve essere di esattamente 11 caratteri.

\item Il \textit{RecapitoTelefonico} di un \textit{Cliente} non può essere lungo meno di 9 caratteri.

\item La \textit{Qualifica} di un \textit{RuoloAziendale} deve essere presente ... ?
\end{itemize}

\subsection{ Vincoli interrelazionali }

\begin{itemize}
\item La \textit{CifraPagata} di un \textit{Pagamento} deve essere $\leq$ dell'\textit{Importo} della \textit{Fattura} a cui si riferisce.

\item Se il titolare di una \textit{Pratica} è un cliente che è una \textit{Organizzazione}, allora la \textit{Pratica} deve essere seguita da almeno un \textit{RuoloAziendale}.
\end{itemize}

 \section{ Schema logico relazionale }

 \subsection{ Formato grafico }

\begin{minipage}{\textwidth}
\textbf{Legenda}:
\includegraphics[width=0.5cm]{ Legenda chiave primaria.png } chiave primaria\hspace{1cm}
\includegraphics[width=0.5cm]{ Legenda chiave esterna.png } chiave esterna\hspace{1cm}
\includegraphics[width=0.5cm]{ Legenda attributo.png } attributo semplice

\begin{center}
\centering 
 \captionof{figure}{Schema logico relazionale}
\centerline{
%\frame{
\includegraphics[width=\textwidth -4cm]{ Schema logico relazionale.png }
}
\end{center}
\end{minipage}

 \subsection{ Formato testuale }

\begin{itemize}
\item \textbf{Pratiche} (\underline{IDpratica}, IDpersona*, IDorganizzazione*)

\item \textbf{Fatture} (\underline{IDfattura}, IDpratica*, IDorganizzazioneEmittente*, IDpersonaIntestataria*, IDorganizzazioneIntestataria*, Importo, DataEmissione)

\item \textbf{Persone} (\underline{IDpersona}, Denominazione, Indirizzo, RecapitoTelefonico, Nome, Cognome, CodiceFiscale)

\item \textbf{Organizzazioni} (\underline{IDorganizzazione}, Denominazione, Indirizzo, RecapitoTelefonico, PartitaIVA)

\item \textbf{Pagamenti} (\underline{IDpagamento}, IDfattura*, IDpersona*, IDorganizzazione*, Modalità, CifraPagata, DataPagamento)

\item \textbf{RuoliAziendali} (\underline{IDruolo}, IDorganizzazione*, Qualifica)

\item \textbf{PersoneRuoliAziendali} (\underline{IDpersona*}, \underline{IDruolo*})

\item \textbf{PraticheRuoliAziendali} (\underline{IDpratica*}, \underline{IDruolo*})
\end{itemize}


 \subsection{ Dipendenze funzionali }

Tutte le relazioni rispettano la forma normale di Boyce-Codd (BCNF), di seguito si riportano le dipendenze funzionali:

\begin{itemize}
\item Pratiche\\È in BCNF perché IDpratica è chiave.
\begin{itemize}
\item IDpratica $\rightarrow$ IDpersona, IDorganizzazione
\end{itemize}

\item Organizzazioni\\È in BCNF perché IDorganizzazione è chiave, e PartitaIVA e RecapitoTelefonico, identificando una singola organizzazione, sono chiavi.
\begin{itemize}
\item IDorganizzazione $\rightarrow$ Denominazione, Indirizzo, RecapitoTelefonico, PartitaIVA
\item PartitaIVA $\rightarrow$ Denominazione, Indirizzo, RecapitoTelefonico, IDorganizzazione
\item RecapitoTelefonico $\rightarrow$ Denominazione, Indirizzo, PartitaIVA, IDorganizzazione
\end{itemize}

\item Persone\\È in BCNF perché IDpersona è chiave, e CodiceFiscale e RecapitoTelefonico, identificando una singola persona, sono chiavi.
\begin{itemize}
\item IDpersona $\rightarrow$ Denominazione, Indirizzo, RecapitoTelefonico, Nome, Cognome, CodiceFiscale
\item CodiceFiscale $\rightarrow$ Denominazione, Indirizzo, RecapitoTelefonico, Nome, Cognome, IDpersona
\item RecapitoTelefonico $\rightarrow$ Denominazione, Indirizzo, Nome, Cognome, CodiceFiscale, IDpersona
\end{itemize}

\item PersoneRuoliAziendali è in BCNF perché non ha dipendenze non banali

\item PraticheRuoliAziendali è in BCNF perché non ha dipendenze non banali

\item RuoliAziendali\\È in BCNF perché IDruolo è chiave
\begin{itemize}
\item IDruolo $\rightarrow$ Qualifica, IDorganizzazione
\end{itemize}

\item Fatture\\È in BCNF perché IDfattura è chiave L'ALTRA È DA CONTROLLAREEEEEEEEEEEEEEEE

\begin{itemize}
\item IDfattura $\rightarrow$ Importo, IDpratica, IDorganizzazioneEmittente, IDpersonaIntestataria, IDorganizzazioneIntestataria, DataEmissione

\item IDpratica $\rightarrow$ IDpersonaIntestataria, IDorganizzazioneIntestataria RICONTROLLAREEEEEEEEEEEEEEEEEEEEEEEEEEE
\end{itemize}

\item Pagamenti\\È in BCNF perché IDpagamento è chiave

\begin{itemize}
\item IDpagamento $\rightarrow$ Modalità, CifraPagata, IDfattura, IDpersona, IDorganizzazione, DataPagamento
\end{itemize}

\end{itemize}

 \section{ Interrogazioni }


 \subsection{ Uso di proiezione, join e restrizione }
Stampa le persone intestatarie di fatture i cui importi sono $> 100$:

\begin{minipage}{\textwidth}
\lstinputlisting[caption=Query 1]{1.sql}
\end{minipage}


 \subsection{ Uso di group by con having, where e sort }

Stampa le modalità di pagamenti in cui CifraPagata $> 10$, con il totale per ogni pagamento $< 10000$ e ordinate in base al totale:

(il senso vorrebbe essere quello di raggruppare i vari pagamenti effettuati in base alle modalità utilizzate e il cui importo superi 10. Esempio: con il bonifico sono state pagate un numero di fatture per un totale di 10000, e poi ordinarle con questo totale)

\begin{minipage}{\textwidth}
\lstinputlisting[caption=Query 2]{2.sql}
\end{minipage}


 \subsection{ Uso di join, group by con having e where }

Stampa il codice fiscale delle persone intestatarie di più di 3 fatture il cui importo è $> 1000$:

\begin{minipage}{\textwidth}
\lstinputlisting[caption=Query 3]{3.sql}
\end{minipage}

 \subsection{ Uso di select annidata con quantificazione esistenziale }

Stampa le organizzazioni che hanno emesso almeno una fattura con un importo $> 100$:

\begin{minipage}{\textwidth}
\lstinputlisting[caption=Query 4]{4.sql}
\end{minipage}

 \subsection{ Uso di select annidata con quantificazione universale }

Stampa le fatture che non sono state pagate in contanti:

\begin{minipage}{\textwidth}
\lstinputlisting[caption=Query 5]{5.sql}
\end{minipage}

 \subsection{ Uso di subquery di confronto quantificato usando una subquery }

Stampa i pagamenti effettuati tramite bonifico il cui importo è maggiore del pagamento massimo effettuato tramite assegno:

\begin{minipage}{\textwidth}
\lstinputlisting[caption=Query 6]{6.sql}
\end{minipage}

 \section{ Piani di accesso }

 \subsection{ Piani di accesso logico }

 \subsubsection{ Query 1) }

 \subsubsection{ Query 2) }

 \subsubsection{ Query 3) }

 \subsection{ Piani di accesso fisico senza uso di indici }

 \subsubsection{ Query 1) }

 \subsubsection{ Query 2) }

 \subsubsection{ Query 3) }

 \subsection{ Piani di accesso fisico con uso di indici }

 \subsubsection{ Query 1) }

 \subsubsection{ Query 2) }

 \subsubsection{ Query 3) }

\end{document}